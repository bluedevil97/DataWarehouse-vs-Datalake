
\documentclass[preprint,12pt]{elsarticle}

\usepackage[spanish]{babel}
\usepackage{amssymb}
\usepackage{graphicx}
\usepackage{lineno}
\usepackage[utf8]{inputenc}
\usepackage{url}
\usepackage{color}
\usepackage{enumerate} 
\usepackage[hidelinks]{hyperref}


\begin{document}
	
	\begin{frontmatter}
		
		
		\title{\huge DataWarehouse vs  Datalake}
		
		\author{Mamani Ayala, Brandon        (2015052715)}
		\author{Quispe Mamani, Angelo	      (2015052826)}
		\author{Vizcarra Llanque, Jhordy	      (2015052719)}
		\author{Ordoñez Quilli, Ronald          (2015052821)}
		\author{Rodriguez Mamani, Juan      (2017057862)}
		
		\address{Tacna, Perú}
		
		\begin{abstract}
			%% Text of we
			
The data warehouses in English take each importance day, as organizations move from schemes of only data collection to schemes of analysis of the same. However, in spite of the great diffusion of the concepts related to data warehouses, there is not too much Information available in Spanish regarding the methodologies fo implement them In this short article we will try to provide a general explanation of one of the most used methodologies. 
		\end{abstract}
\end{frontmatter}
%%

	
	%%
	%\linenumbers
	
	%% main text
	\section{Resumen}
Los almacenes de datos (data warehouses en inglés) toman cada día mayor importancia, a medida que las organizaciones pasan de esquemas de sólo recolección de datos a esquemas de análisis de los mismos. Sin embargo a pesar de la gran difusión de los conceptos relacionados con los almacenes de datos, no existe demasiada información disponible en castellano en cuanto a las metodologías para implementarlos. En este breve artículo intentaremos brindar una explicación general de una de las metodologías más usadas \\
	%%
	
	%%
	%\linenumbers
	
	%% main text



	%%
	
	%%
	%\linenumbers
	
	%% main text
\section{Introduccion}
\section{DataWarehouse}

\subsection{Materiales y Metodos}

\subsection{Resultados}

\subsection{Conclusiones}

\begin{itemize}
	\item Las particiones no se procesaban en paralelo si no secuencialmente, lo que hace que sea más lento el procesamiento.
	\item No se pueden usar multiples idiomas.
	\item Si son muchos datos tarda bastante en manejar configuraciones de diferentes particiones.
	\item El modelo tabular acapara demasiada memoria RAM y a su vez es dependiente de tal que afectará a otras aplicaciones.
\end{itemize}

\section{Datalake}

\subsection{Materiales y Metodos}

\subsection{Resultados}

\subsection{Conclusiones}

\begin{itemize}
	\item Las particiones no se procesaban en paralelo si no secuencialmente, lo que hace que sea más lento el procesamiento.
	\item No se pueden usar multiples idiomas.
	\item Si son muchos datos tarda bastante en manejar configuraciones de diferentes particiones.
	\item El modelo tabular acapara demasiada memoria RAM y a su vez es dependiente de tal que afectará a otras aplicaciones.
\end{itemize}



%%
	
	%%
	%\linenumbers
	
	%% main text

	
	\newpage
	
		%ESTILO
	%ARCHIVO .bib
	   \begin{thebibliography}{0}
              \bibitem{Juan} https://bib.irb.hr/datoteka/102195.t09r02.pdf 
                 \bibitem{Juan} https://www.sarjen.com/2016/03/15/what-are-the-pros-and-cons-of-tabular-model-over-multi-dimension-cube-and-relation-database/
                 \bibitem{Juan} https://www.element61.be/en/resource/choice-between-tabular-or-multidimensional-models-sql-server-analysis-services-2012
                  \bibitem{Jhordy} https://docs.microsoft.com/es-es/sql/analysis-services/comparing-tabular-and-multidimensional-solutions-ssas?view=sql-server-2017
 \bibitem{Jhordy} https://www.businessintelligence.info/definiciones/que-es-modelo-dimensional.html
                    \bibitem{Brandon} https://www.businessintelligence.info/definiciones/que-es-modelo-dimensional.html


         \end{thebibliography}
	
\end{document}

%%
%% End of file `elsarticle-template-1-num.tex'.
